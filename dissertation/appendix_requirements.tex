\chapter{Detailed Requirements}\label{appendix_requirements}

\section{Initial Requirements}\label{initial-requirements}

This project will look at augmenting 3D rendering software to produce
{[}N10: high-quality{]} {[}N20: scientific{]} {[}N30: 3D{]} surfaces
with {[}N40: pseudo hand drawn{]} appearance. The project will focus on
extending the {[}N50: Blender {]} {[}blender.org{]} renderer to produce
these graphs, most likely via {[}N60: Python{]} scripting.

Traditional hand drawn plots (see below) are able to {[}N70: reveal
structure{]} in 3D surfaces that is often lost in modern renders.
Although modern renders have accurate light transport models, they are
designed for photo realism rather than to reveal the structure of
surfaces. This is particularly relevant when producing figures for
{[}N80: reproduction{]}, which must be {[}N90: clear {]} and might only
be {[}N100: monochrome{]}.

Traditional artists developed techniques to {[}N110: reveal shading{]}
and surface features. The aim of this project will be to develop a
system to produce high quality images {[}N120: automatically{]},
according to a specification provided by a user (e.g. {[}F10:
line-only{]}, {[}F20: highlight creases{]}, {[}F30: no lighting{]}).

\begin{itemize}
\tightlist
\item
  A {[}N130: blender add-on{]} is to be developed which will produce
  images of hand-drawn appearance.
\item
  {[}N140: Animation will not be supported{]}, i.e. {[}N150: temporal
  cohesion is not a concern{]}.
\item
  {[}N160: 3D model will be the input{]}.
\item
  {[}N170: Vector SVG will be the output{]}.
\item
  Drawing style shall reveal surface details and shape, {[}F40: regions
  of high-curvature{]} etc.
\item
  {[}N180: A specific drawing style shall be chosen{]}, although
  {[}N190: system design shall be flexible enough to add additional
  styles{]} at a later date.
\item
  {[}N200: No requirement to reveal surface texture{]}.
\item
  {[}F50: Lines shall be scaled according to distance of the camera from
  the object{]}.
\item
  The approach is assumed to require Python scripting to {[}N210:
  process render layers produced by the blender rendering engine{]}.
\item
  A nice feature to have could be to {[}F60: mark areas of the model
  which must be rendered with a line/pattern{]} (e.g.~to draw attention
  to specific areas of interest, or to allow non-deterministic output).
\item
  User interaction will be via the existing Blender GUI. {[}N220: Custom
  Blender GUI panels shall be developed as required to support all
  functionality.{]}
\item
  Focus is on black and white images, however it may be useful to look
  at {[}F70: limited use of colour{]} (2-3 colours max).
\end{itemize}

\section{Questionnaire Responses}\label{questionnaire-responses}

\subsection{Non-Functional Requirements}\label{non-functional-requirements}

\subsubsection{Clarification of Initial Requirements}\label{clarification-of-initial-requirements}

\begin{itemize}
\tightlist
\item
  C10. Ref N10, define the term ``high-quality''.

  \begin{itemize}
  \tightlist
  \item
    {[}N230: Parsimony of ink: draw only the strokes which are needed
    and not superfluous ones.{]} Strokes should reveal structure.
  \item
    {[}N240: Accuracy of curvature: strokes should not be excessively
    smoothed, nor excessively rough and should reflect the underlying
    geometry.{]}
  \item
    {[}N250: Consistency of lines: lines should not be unnaturally
    broken or incorrectly merged.{]} {[}N260: Strokes should continue
    together cleanly.{]}
  \item
    {[}N270: Accuracy of joints: points where strokes merge
    (e.g.~corners) should be accurately aligned, not offset messy
    crossings.{]}
  \item
    {[}N280: Variation of strokes: stroke thickness should vary in a way
    that reveals structure, e.g.~lighting or perspective
    foreshortening.{]}
  \item
    {[}N290: Consistency of reproduction: small changes in mesh or view
    should result in small apparent changes in drawing (but temporal
    coherency, as for animations, is not required; randomness is quite
    acceptable).{]}
  \end{itemize}
\item
  C20. Ref N50, state the minimum Blender version number to support.

  \begin{itemize}
  \tightlist
  \item
    Blender 2.79
  \end{itemize}
\item
  C30. Ref N100, interpretation is ``monochrome'' means strokes will be
  rendered in a single colour, rather than in tones of a single colour.
  Please clarify if otherwise.

  \begin{itemize}
  \tightlist
  \item
    Primary goal will be B/W rendition (i.e.~solid black strokes on
    white background). Optionally, {[}F80: strokes of a number of other
    shades may be included as an optional requirement, for example to
    indicate lighting of different colours on different facets.{]}
  \end{itemize}
\item
  C40. Ref N120 and N160, interpretation is the the User will configure
  a Blender scene with a 3D surface mesh, including creation and
  positioning of a camera and any required lighting. This will be the
  starting point for interaction with the System. Please clarify if
  otherwise.

  \begin{itemize}
  \tightlist
  \item
    User will create the Blender scene. Software will be a pass after
    the scene has been completely created (modelled, lit, camera view
    set, etc.).
  \end{itemize}
\item
  C50. Ref N170, state the minimum SVG version number to support.

  \begin{itemize}
  \tightlist
  \item
    SVG 1.1.
  \end{itemize}
\item
  C60. Ref N180, development will focus on producing stroke-based
  illustrations in the Pen-and-Ink style, which aligns with requirements
  N10, N20, N30, N40, N70, N80, N90, N100, N110, F10, F20, F30 and F40.
  If another illustration style is thought to be better suited, please
  state it here.

  \begin{itemize}
  \tightlist
  \item
    Stroke based rendering is the priority. Other rendering styles might
    include cross-hatching.
  \end{itemize}
\item
  C70. Ref N220, if there is a preference for where the GUI panels
  should be located within the Blender interface, please state so here.

  \begin{itemize}
  \tightlist
  \item
    No strong preference. Probably as a new panel in the right hand
    panels, or perhaps at the bottom of the render panel?
  \end{itemize}
\end{itemize}

\subsubsection{Additional Requirements}\label{additional-requirements}

\begin{itemize}
\tightlist
\item
  N300. Which operating systems shall be supported? Please also indicate
  minimum version numbers.

  \begin{itemize}
  \tightlist
  \item
    Windows 10 support desired and should be tested; Ubuntu Linux as a
    lower priority; ideally no OS-specific requirements.
  \end{itemize}
\item
  N310. Please rate the relative importance of each of the following
  characteristics:

  \begin{itemize}
  \tightlist
  \item
    Functionality.
  \item
    Reliability.
  \item
    Usability.
  \item
    Portability.
  \item
    Efficiency.
  \item
    Maintainability.
  \end{itemize}
\item
  N320. In a few sentences, state the most critical measure of success
  for the System, i.e.~what does a successful product look like?

  \begin{itemize}
  \tightlist
  \item
    A user opens a Blender scene. They click a button and an SVG
    document is produced that produces an aesthetically pleasing,
    accurate representation of the scene, using only strokes, with a
    parsimony of ink and effective communication of the surface
    features.
  \end{itemize}
\item
  N330. It is assumed that only a single mesh will be present in the
  Scene to be rendered. Please clarify if otherwise.

  \begin{itemize}
  \tightlist
  \item
    Single mesh is reasonable assumption.
  \end{itemize}
\item
  N340. If there are other non-functional requirements not captured by
  the sections above, please state them here.

  \begin{itemize}
  \tightlist
  \item
    None I can think of.
  \end{itemize}
\end{itemize}

\subsection{Functional Requirements}\label{functional-requirements}

\subsubsection{Clarification of Initial
Requirements}\label{clarification-of-initial-requirements-1}

\begin{itemize}
\tightlist
\item
  C80. Ref F10, interpretation ``line-only'' means only the object
  outline/silhouette will be rendered. Please clarify if otherwise.

  \begin{itemize}
  \tightlist
  \item
    Line-only could include synthesised lines e.g.~to show contours
    which would not fall on the silhouette.
  \end{itemize}
\item
  C90. Ref F20, interpretation is ``highlight creases'' means only edges
  whose neighbouring faces meet at an angle greater than a User- defined
  value will be rendered. Please clarify if otherwise.

  \begin{itemize}
  \tightlist
  \item
    Creases would be defined in terms of curvature (either the geometric
    curvature of the mesh, or the visual apparent curvature) rather than
    in terms of angles.
  \end{itemize}
\item
  C100. Ref F30, interpretation is ``no lighting'' means only geometric
  data (or world data such as ambient occlusion) would be used to
  determine the placement of feature-highlighting strokes. Please
  clarify if otherwise.

  \begin{itemize}
  \tightlist
  \item
    Actually I think lighting is fairly important and should be
    included. In particular {[}F90: it should be possible to adjust
    density or thickness of strokes according to incident light, either
    diffuse direct light, or to emphasise areas using AO information.{]}
  \end{itemize}
\item
  C110. Ref F50, should the User have the option to toggle this stroke
  tapering effect on and off?

  \begin{itemize}
  \tightlist
  \item
    User should be able to control stroke effects.
  \end{itemize}
\item
  C120. Ref F50, should the User have the option to influence degree of
  stroke taper by controlling max and min stroke thickness?

  \begin{itemize}
  \tightlist
  \item
    Yes.
  \end{itemize}
\item
  C130. Ref F60, selection of faces or edges will either be via the
  existing selection tools available in the Blender wireframe view, or
  by ``painting'' areas using the Blender Grease Pencil. If one
  particular method is more desirable, or if another method is required,
  please state it here.

  \begin{itemize}
  \tightlist
  \item
    Would expect either Grease Pencil or the ``Mark Seam'' functionality
    in the mesh editor (used for indicating UV unwrapping). As a general
    point, {[}N350: might want to consider the UV maps as a useful
    source of information -- this is a way the user can specify the
    structure of the shape in more detail.{]}
  \end{itemize}
\item
  F100. Should there be natural variation in generated stroke
  waviness/curvature?

  \begin{itemize}
  \tightlist
  \item
    Natural variation as an optional requirement, configurable if
    present.
  \end{itemize}
\item
  F110. If yes to the above, should the User have control over the
  global variation of waviness/curvature?

  \begin{itemize}
  \tightlist
  \item
    Natural variation as an optional requirement, configurable if
    present.
  \end{itemize}
\item
  F120. Should there be natural variation in generated stroke thickness
  along the length of a stroke (independent of distance from the
  camera)?

  \begin{itemize}
  \tightlist
  \item
    Yes, but optional requirement, and configurable.
  \end{itemize}
\item
  F130. If yes to the above, should the User have control over the
  global variation of thickness?

  \begin{itemize}
  \tightlist
  \item
    Yes, but optional requirement, and configurable.
  \end{itemize}
\item
  F140. Should the User have control over the preferred global density
  of generated strokes?

  \begin{itemize}
  \tightlist
  \item
    Yes, configuration of density is important. Perhaps only globally;
    ideally also locally somehow (but may not be feasible).
  \end{itemize}
\item
  F150. Should the User have control over the directionality of
  generated strokes? If so, how should this be controlled?

  \begin{itemize}
  \tightlist
  \item
    Not clear how this could be usefully controlled? Doesn't seem a key
    requirement.
  \end{itemize}
\item
  F160. Should the User have control over the global stroke colour?

  \begin{itemize}
  \tightlist
  \item
    Yes, though this isn't critical (since could just open in an SVG
    editor and recolour).
  \end{itemize}
\item
  F170. Should the User have control over the canvas (image background)
  colour?

  \begin{itemize}
  \tightlist
  \item
    Yes, though this isn't critical (since could just open in an SVG
    editor and recolour).
  \end{itemize}
\item
  F180. If there are other functional requirements not captured by the
  sections above, please state them here.

  \begin{itemize}
  \tightlist
  \item
    If supporting multi-meshes, would be good to be able to {[}N360:
    separate strokes from objects into SVG layers so they can be edited
    separately.{]} {[}N370: Separating into layers could also be used
    for strokes from different sources (e.g.~silhouette, contour,
    strokes from different light sources).{]} This isn't a critical
    feature, but being able to easily work with groups of strokes in the
    SVG editing stage would be useful. Certainly, one preference would
    be that {[}N380: editing the generated SVG files should be
    structured so that this is pleasant and efficient to do.{]} {[}N390:
    Consideration of how shadows will be handled is important.{]} This
    might be straightforward but should be considered.
  \end{itemize}
\end{itemize}

\section{Assumptions}\label{assumptions}

\begin{itemize}
\tightlist
\item
  A purely {[}A10: image-space{]} based approach will be taken, i.e.~no
  inputs are taken from Blender's object-space.
\item
  {[}A20: It is not possible to obtain image-space data directly from
  Blender's render passes via the Python API {]}.
\item
  As a consequence of A20, {[}A30: Blender must save the required input
  images to disk before they can be processed by the System.{]}
\item
  As a consequence of A30, we define a new requirement: {[}F190: The
  System shall automatically activate required render passes and produce
  the required compositor node setup for saving these images to disk.{]}
\item
  {[}A40: The goal of the System is to obtain the Final Rendering for
  use in other software packages.{]}
\item
  As a consequence of A40, we define a new requirement: {[}F200: The
  Final Rendering shall be saved to disk, and will not be presented
  on-screen within Blender.{]}
\end{itemize}
