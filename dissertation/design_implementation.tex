% 4-5 pages total.

% Goals:
% System architecture major design decisions and rationale.
% Implementation details (~1 page).
% Screenshots or other similar things (~1 page).
% Additional details; eg design diagrams in the Appendix.

% Marking:
% Is the software product well-designed, functional, reliable, robust, efficient, usable, maintainable, and well-documented? Has it been demonstrated?
% A-band: The software product is extremely well designed, implemented, and documented.

\chapter{Design \& Implementation}\label{design_implementation}

% - Design for blender plugin (multifile).
% - Design considerations for testing (brief, cover testing in next section).
% - Separation of blender-specific from backend.
% - Levels of abstraction between modules.

% - Logging.

The final system architecture diagram is presented in Appendix \ref{appendix_design}.

A key requirement is for the System to be integrated with Blender as an add-on. 
Blender's add-on infrastructure has certain expectations\footnote{\url{https://wiki.blender.org/wiki/Process/Addons/Guidelines}} which govern how a potential add-on should be structured, and these aspects had to be considered at an early design stage.

This presented an initial challenge - although Blender's API documentation provides guidance\footnote{\url{https://docs.blender.org/manual/en/dev/advanced/scripting/addon_tutorial.html}} applicable to trivial (single-file) add-ons, there is limited information available for structuring more advanced multi-file add-ons.
